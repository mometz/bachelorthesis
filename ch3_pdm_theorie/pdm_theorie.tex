\chapter{Predictive Maintenance}\label{ch:pdm_theorie}
Die zunehmende Komplexität und Vernetzung moderner Systeme erfordert effizientere Wartungsstrategien. Predictive Maintenance nutzt die
enormen Datenmengen solcher Systeme, um drohende Defekte frühzeitig zu erkennen und Ausfälle zu verhindern. Dadurch können
Ausfallzeiten minimiert und die Lebensdauer der Systeme verlängert werden.

Nach dem ISO 13379{-}1 Standard~\cite{ISO2012} wird Predictive Maintenance als Wartungsstrategie definiert, die die Zustandsüberwachung
eines Systems oder einer Komponente nutzt, um Zustandsänderungen zu detektieren, die auf bevorstehende Ausfälle hinweisen können.
Eine passende Analogie findet sich in der Medizin: Wenn der menschliche Körper Anzeichen einer bevorstehenden Krankheit zeigt, können
diese Symptome vom Arzt genutzt werden, um eine Diagnose zu stellen und entsprechende Maßnahmen zu ergreifen.
Diese Zustandsüberwachung erlaubt es, dass Wartungsarbeiten zu einem Zeitpunkt stattfinden können, der für alle Beteiligte passend ist
und minimale Einschnitte in Produktions- oder Prozesslaufzeiten bedeutet~\cite{Scheffer2004}.

Im Vergleich zu traditionellen Ansätzen bietet Predictive Maintenance signifikante Vorteile,
doch auch die klassischen Methoden haben ihre Berechtigungen. Diese werden im Folgenden diskutiert.

\section{Traditionelle Wartungsansätze}\label{sec:trad_maintenance}
Zu den traditionellen Wartungsansätzen gehören die reaktive und die präventive Wartung. Bei der reaktiven oder auch korrektiven
Wartung wird, gemäß der Namensgebung, erst bei vollständigem Ausfall von Komponenten gehandelt.
Der Vorteil dieser Methode liegt in der sehr geringen Planung und Überwachung. Für Komponenten,
die nicht kritisch oder essenziell sind, oder die ein sehr geringes Ausfallrisiko aufweisen, kann die reaktive Wartung sinnvoll sein.
Für alle anderen Bestandteile bzw.~die Gesamtheit eines Systems ist sie jedoch höchst ineffizient, da Wartungsmaßnahmen erst dann
veranlasst und geplant werden, wenn das System ausfällt. Zudem gestaltet sich die Diagnose dann auch als potenziell schwierig, da die
Fehlerquelle noch gefunden werden muss~\cite{Abdelli2022}.

Demgegenüber steht die präventive Wartung, die Maßnahmen am Ende eines vorher festgelegten Zeitintervalls oder nach Ablauf einer
bestimmten Betriebsdauer festlegt. Üblicherweise orientieren sich diese Wartungsintervalle entweder an der MTTF (\textit{Mean Time
To Failure}) oder an der Badewannenkurve~\cite{Andrews2002}. Dieser Ansatz hat gewisse Vorteile für Komponenten, die nicht im
dauerhaften Betrieb sind, sofern ausreichend geschultes Personal vorhanden ist mit genügend Zeit, die Wartungsarbeiten durchzuführen.
Nachteile liegen im potenziell schlechten Timing der Wartungsarbeiten, die entweder zu früh oder zu spät stattfinden. Ohne Überwachung
des Systemzustands ist schwer absehbar, in welchem Stadium seiner Lebensdauer sich ein System befindet, und Komponenten, die noch
eine gewisse Zeit weiterlaufen könnten, werden zu früh ausgetauscht. Durch die Wartung verkürzt sich auch die Betriebsdauer des ganzen
Systems und verursacht vermeidbare Kosten~\cite{Scheffer2004}.
