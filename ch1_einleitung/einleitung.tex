\chapter{Einleitung und Motivation}
In modernen Verkehrssystemen ist die Sicherstellung von Zuverlässigkeit, Sicherheit und Verfügbarkeit essenziell,
da Störungen nicht nur hohe wirtschaftliche Kosten, sondern auch erhebliche Risiken für die Funktionsfähigkeit
kritischer Infrastrukturen mit sich bringen. Dies macht effektive Wartungsstrategien zu einem unverzichtbaren
Bestandteil der Instandhaltungsplanung zur Einhaltung der genannten Gütefaktoren.

Dabei gibt es mehrere Ansätze, die historisch gewachsen sind. Die reaktive bzw.~kurative Wartung greift dann ein,
wenn Fehler auftreten oder Komponenten nicht mehr ihre Funktionalität erfüllen können. Dies hat den offensichtlichen
Nachteil, dass erst durch Systemausfälle der Wartungseinsatz herbeigeführt wird. Dementsprechend zieht dies einen
finanziellen Nachteil mit sich und sorgt für Ausfallzeiten im System. Zudem gibt es den Ansatz der präventiven oder
auch zeitbasierten Wartung. Hier erfolgt der Wartungseinsatz auf Grundlage von festgelegten Wartungsintervallen, die entweder auf einem
Zeitplan oder einer Betriebsdauer basieren. Diese finden also unabhängig vom tatsächlichen Zustand des Systems statt
und können dementsprechend leicht umgesetzt werden, aber gleichzeitig überflüssige Wartungsarbeiten verursachen, sollte das
System noch in einem guten Zustand sein. Unerwartete oder plötzlich eintretende Ausfälle können so auch nicht
verhindert werden.

% Traditionelle Ansätze wie reaktive Wartung, die erst bei Fehlern eingreift,
% oder präventive Wartung, die fixe Intervalle einhält, stoßen jedoch an ihre Grenzen. Predictive Maintenance bietet
% hier eine innovative Alternative, indem sie den Systemzustand auf Grundlage von Echtzeit- und historischen Daten
% vorhersagt und bedarfsgerechte Wartungsmaßnahmen ermöglicht.
% 
% Verkehrsinfrastrukturen und Indusatriesysteme im Allgmeinen werden durch ihre wachsende Komplexität und die
% Integration neuer Technologien zunehmend anfälliger für Ausfälle. Mit zunehmendem Alter wird ihre Instandhaltung
% nicht nur kostenintensiver, sondern auch schwieriger. Die Überwachung signifikanter Systemparameter ermöglicht zwar in einem
% zustandsbasierten Wartungsansatz (Condition-Based Maintenance) gezielte Eingriffe, stößt jedoch an Grenzen, wenn
% Ressourcen oder Ersatzteile nicht verfügbar sind.

Verkehrsinfrastrukturen und auch Industriesysteme im Allgemein werden durch ihre wachsende Komplexität und die Integration neuer
Technologien zunehmend anfälliger für Ausfälle und Defekte. Mit steigendem Alter wird ihre Instandhaltung nicht nur kostenintensiver,
sondern auch schwieriger. Diese traditionellen Wartungsansätze stoßen jedoch in komplexen und zunehmend vernetzten Systemen an ihre
Grenzen. Um der steigenden Komplexität und den damit verbundenen Herausforderungen in der Wartung zu begegnen, ist es notwendig,
fortschrittlichere Methoden zu integrieren, die präzisere Vorhersagen über den Zustand von Systemen ermöglichen.

Die prädiktive Wartung bzw.~\textbf{Predictive Maintenance} setzt an dieser Stelle an. Sie basiert auf der Analyse
von Echtzeitdaten, die aus einer Vielzahl von Sensoren entnommen werden können. Maschinelles Lernen stellt sich hier als ideale
Herangehensweise heraus, um sich anbahnende Fehler und Ausfälle vorherzusagen. Wartungsmaßnahmen werden
also erst dann ergriffen, wenn der Systemzustand auf einen bevorstehenden Ausfall hindeutet. So können Ressourcen effizient
genutzt und Ausfallzeiten minimiert werden~\cite{TobonMejia2012}.

Durch immer mehr verfügbare Mess- und Leistungsdaten steht eine entsprechend große Datenmenge zur Analyse bereit, die zur
Zustandsfeststellung und -vorhersage herangezogen werden kann. Diese Datensätze werden über einen OPC UA Server zur
Verfügung gestellt. OPC UA steht für \textit{Open Platform Communication Unified Architecture} und ist ein Industriestandard
für Datenaustausch sowie horizontale und vertikale Kommunikation. Ein großer Vorteil von OPC UA besteht in seiner
Herstellerunabhängigkeit durch semantische Interoperabilität sowie seiner Vielzahl an Sicherheitsmechanismen~\cite{Babel2024}.
Die Datenerfassung erfolgt lokal auf jedem Gerät, welches im Zuge dessen auch einen eigenen Server betreibt, um die aufgenommenen
Daten bereitzustellen. Auf diese Weise wird eine dezentrale Sammlung relevanter Parameter wie Temperatur,
Leistung und Systemauslastung ermöglicht. Die kontinuierliche Überwachung und Speicherung dieser Daten in einer Datenbank
bildet die Grundlage für die Entwicklung und Anwendung prädiktiver Wartungsstrategien.

Ein zentraler Aspekt dabei ist die Detektion von Anomalien, die als Abweichungen vom normalen Verhalten definiert
werden~\cite{Chandola2009}. In multivariaten Systemen erschwert die hohe Dimensionalität und Komplexität der Daten ihre Erkennung.
Unüberwachte maschinelle Lernverfahren bzw.~\textbf{Unsupervised Learning} erweisen sich hier als besonders vielversprechend,
da sie Muster und Zusammenhänge in hochdimensionalen, unstrukturierten Datensätzen identifizieren können, ohne auf eine vorherige
Datenkennzeichnung angewiesen zu sein.

Ziel dieser Arbeit ist es, Unsupervised Learning Algorithmen für die Erkennung und Interpretation von Anomalien in
heterogenen Datensätzen eines Systems in der Verkehrstechnik zu nutzen und damit die Effizienz von Predictive Maintenance
in der Verkehrstechnik zu optimieren. Dies soll durch eine frühzeitige Identifikation potenzieller Fehler erreicht werden.

Dabei wird zunächst in~\hyperref[ch:theorie]{Kap.~\ref*{ch:theorie}} ein theoretischer Hintergrund über die Grundlagen
der Anomalieerkennung, Predictive Maintenance und verschiedener Machine Learning Ansätze gegeben und warum sich Unsupervised
Learning optimal für die hier genannten Zwecke eignet.
