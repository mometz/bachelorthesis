\chapter{Einleitung und Motivation}
Vitronic Mautbrücken sind aufgrund ihrer schweren Erreichbarkeit und der Tatsache, dass sie über Autobahnen installiert
sind, eine besondere Herausforderung im Bereich der Wartung. Jede Inspektion oder Reparatur erfordert nicht nur die
Bereitstellung spezialisierter Techniker und Geräte, sondern auch eine präzise Planung, um Verkehrsstörungen zu
minimieren und die Sicherheit sowohl der Wartungsteams als auch der Verkehrsteilnehmer zu gewährleisten.

Ein technischer Defekt erfordert häufig eine komplexe Koordination mehrerer Teams und führt zu Verkehrsstörungen. Dadurch
entsteht die Notwendigkeit, Wartungseinsätze effizienter zu gestalten, um die Funktionalität der Infrastruktur und die
Sicherheit der Verkehrsteilnehmenden sowie der Servicemitarbeitenden zu gewährleisten. Gleichzeitig wächst der Druck, diese
Einsätze planbarer und seltener durchzuführen. Genau hier setzt die Motivation dieser Arbeit an: neue Methoden zu entwickeln, die
eine präzise Überwachung des Zustands der Brücken ermöglichen, ohne ständige physische Präsenz vor Ort.

Diese Arbeit befasst sich konkret mit der \textbf{Smart Sensor Platform X1}~-~kurz \textbf{SSPX1}, einem System zur
Erfassung, Verarbeitung und Weiterleitung von Bilddaten. Die vorhandenen Komponenten zur Mautüberwachung umfassen zwei Kameras,
ein eingebettetes High-Performance-Computermodul mit GPU-Unterstützung, ein High-Power-LED-Blitzmodul sowie
eine Stromversorgungshardware mit Weitbereichsspannungseingang. Die SSPX1 nutzt zahlreiche Sensordaten zur
kontinuierlichen und sorgfältigen Überwachung des Systemzustands.

\textbf{Predictive Maintenance} nutzt diese Daten, um Ausfälle vorherzusagen. Sie basiert auf der Analyse
von Echtzeitdaten, die aus einer Vielzahl von Sensoren entnommen werden können. Maschinelles Lernen bietet sich hier als effektive
Methode an, um sich anbahnende Fehler und Ausfälle vorherzusagen und abzufangen. So können Ressourcen effizient
genutzt und Ausfallzeiten minimiert werden~\cite{TobonMejia2012}.

Diese großen Datensätze werden über einen OPC UA Server zur Verfügung gestellt. OPC UA steht für \textit{Open Platform Communication
Unified Architecture} und ist ein Industriestandard für Datenaustausch sowie horizontale und vertikale Kommunikation, unter Anderem
aufgrund seiner Herstellerunabhängigkeit~\cite{Babel2024}.
Die Datenerfassung erfolgt lokal auf jeder SSPX1, welche im Zuge dessen auch einen eigenen Server betreibt, um die aufgenommenen
Daten bereitzustellen. Auf diese Weise wird eine dezentrale Sammlung relevanter Parameter wie Temperatur,
Leistung und Systemauslastung ermöglicht. Die kontinuierliche Überwachung und Speicherung dieser Daten in einer Datenbank
bildet die Grundlage für die Entwicklung und Anwendung prädiktiver Wartungsstrategien.

Ein zentraler Aspekt dabei ist die Detektion von Anomalien, die als Abweichungen vom normalen Verhalten definiert
werden~\cite{Chandola2009}. In multivariaten Systemen erschwert die hohe Dimensionalität und Komplexität der Daten ihre Erkennung.
Unüberwachte maschinelle Lernverfahren bzw.~\textbf{Unsupervised Learning} erweisen sich hier als besonders vielversprechend,
da sie Muster und Zusammenhänge in hochdimensionalen, unstrukturierten Datensätzen identifizieren können, ohne auf eine vorherige
Datenkennzeichnung angewiesen zu sein~\cite{Chandola2009, Wenig2024}.

Ziel dieser Arbeit ist es, Unsupervised Learning Algorithmen für die Erkennung und Interpretation von Anomalien in
heterogenen Datensätzen eines Systems in der Verkehrstechnik zu nutzen und damit die Effizienz von Predictive Maintenance
in der Verkehrstechnik zu optimieren. Dies soll durch eine frühzeitige Identifikation potenzieller Fehler erreicht werden.

Dabei wird zunächst in~\hyperref[ch:pdm_theorie]{Kap.~\ref*{ch:pdm_theorie}} ein theoretischer Hintergrund über die Grundlagen
von Predictive Maintenance und verschiedener Machine Learning Ansätze gegeben und warum sich Unsupervised Learning optimal für
die hier genannten Zwecke eignet.
% weitere Kapitel vorstellen ...
