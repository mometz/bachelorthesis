\chapter{Einleitung und Motivation}
In modernen Verkehrssystemen spielen zuverlässige und effiziente Wartungsstrategien eine entscheidende Rolle. Störungen oder Ausfälle können nicht nur hohe Kosten verursachen,
sondern auch die Sicherheit und Verfügbarkeit kritischer Infrastrukturen beeinträchtigen. Traditionelle Ansätze, wie die reaktive oder präventive Wartung, stoßen an ihre Grenzen,
da sie entweder zu spät eingreifen oder unnötige Wartungsarbeiten auslösen können.

Predictive Maintenance bietet eine Lösung, indem es basierend auf Daten Vorhersagen über den Zustand von Systemen trifft und Wartungsmaßnahmen gezielt plant. Besonders vielversprechend
ist dabei der Einsatz von unüberwachtem maschinellem Lernen zur Anomaliedetektion und Mustererkennung. Diese Methoden erlauben es, ohne vorherige Labeling-Prozesse Anomalien in komplexen
und multivariaten Datensätzen zu identifizieren \cite{Wenig2024}.

Die Verkehrstechnik stellt dabei eine besondere Herausforderung dar, da Daten oft aus heterogenen Quellen stammen, wie Sensordaten, Kamerabildern oder Systemprotokollen, und häufig großen Schwankungen unterworfen sind. Unsupervised Learning kann hier eine Schlüsselrolle spielen, um versteckte Zusammenhänge in den Daten aufzudecken und Anomalien zu klassifizieren.

Die zentrale Fragestellung dieser Arbeit lautet daher:
Wie können unüberwachte Lernmethoden genutzt werden, um Anomalien zu erkennen und damit Predictive Maintenance in der Verkehrstechnik effizient zu gestalten?