\chapter{Predictive Maintenance}\label{ch:pdm_theorie}
Die Notwendigkeit von Predictive Maintenance wird immer dringlicher, da viele moderne Systeme zunehmend komplex und hochvernetzt sind.
Die Systeme erzeugen eine enorme Menge an Daten, die wertvolle Informationen über den Zustand des
Systems liefern können. Predictive Maintenance nutzt diese Daten, um Ausfälle und Defekte vorherzusagen. So wird eine frühzeitige Intervention ermöglicht, die Ausfallzeiten minimiert und die Lebensdauer eines Systems verlängert.
Jedoch haben alle Wartungsansätze ihre Daseinsberechtigung, aber auch Nachteile. Diese werden im Folgenden diskutiert.

\section{Traditionelle Wartungsansätze}
Zu den traditionellen Wartungsansätzen gehören die reaktive und die präventive Wartung. Bei der reaktiven oder auch korrektiven
Wartung wird, gemäß der Namensgebung, erst dann gehandelt, wenn Fehler auftreten und Komponenten gänzlich ausfallen und nicht mehr
funktionstüchtig sind. Der Vorteil dieser Methode liegt in der sehr geringen Planung und Überwachung. Für Komponenten,
die nicht kritisch oder essenziell sind, oder die ein sehr geringes Ausfallrisiko aufweisen, kann die reaktive Wartung sinnvoll sein.
Für alle anderen Bestandteile bzw.~die Gesamtheit eines Systems ist sie jedoch höchst ineffizient, da Wartungsmaßnahmen erst dann
veranlasst und geplant werden, wenn das System ausfällt. Zudem gestaltet sich die Diagnose dann auch als potenziell schwierig, da die
Fehlerquelle noch gefunden werden muss~\cite{Abdelli2022}.

Demgegenüber steht die präventive Wartung, die Maßnahmen am Ende eines vorher festgelegten Zeitintervalls oder nach Ablauf einer
bestimmten Betriebsdauer festlegt. Üblicherweise orientieren sich diese
Wartungsintervalle entweder an der MTTF (\textit{Mean Time To Failure}) oder an der Badewannenkurve~\cite{Andrews2002}.
Dieser Ansatz hat gewisse Vorteile für Komponenten, die nicht im dauerhaften
Betrieb sind, sofern ausreichend geschultes Personal vorhanden ist mit genügend Zeit, die Wartungsarbeiten durchzuführen.
Nachteile liegen im potenziell schlechten Timing der Wartungsarbeiten, die entweder zu früh oder zu spät stattfinden. Ohne Überwachung
des Systemzustands ist schwer absehbar, in welchem Stadium seiner Lebensdauer sich ein System befindet, und Komponenten, die noch
eine gewisse Zeit weiterlaufen könnten, werden zu früh ausgetauscht. Durch die Wartung verkürzt sich auch die Betriebsdauer des ganzen
Systems und verursacht vermeidbare Kosten~\cite{Scheffer2004}.
