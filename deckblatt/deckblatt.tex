%%%%%%%%%%%%%%%%%%%%%%%%%%%%%%%%%%%%%%%%%%%%
% deckblatt-BT-ET-IT-ger-v20170301.tex
% Deckblatt 
% Bachelorstudiengang Elektrotechnik
% Studienrichtung Elektrotechnik & Informationstechnik
%
% author:        		K.H. Hofmann 
% created:       	07.08.2014
% last revision:	01.03.2017
%%%%%%%%%%%%%%%%%%%%%%%%%%%%%%%%%%%%%%%%%%%%

% Page border should be:
% top: 15 mm
% oddside:  left 25 mm, right 15 mm
% evenside: left 15 mm, right 25 mm
%
% Eingaben
\newcommand{\Bearbeiter}{Moritz Mühlhausen}
\newcommand{\Thema}{Predictive Maintenance mithilfe von Unsupervised Learning in der Verkehrstechnik}
\newcommand{\Referent}{Prof.\ Dr.-Ing. Michael Voigt}
\newcommand{\Korreferent}{M. Sc. Felix Becker, Vitronic Machine Vision GmbH}
%%%%%%%%%%%%%%%%%%%%%%%%%%%%%%%%%%%%%%%%%%%%
%\begin{document}
%%%%%%%%%%%%%%%%%%%%%%%%%%%%%%%%%%%%%%%%%%%%
% Deckblatt 1. Seite 
\thispagestyle{empty} 
%
\vspace{-20mm}
\begin{minipage}[t]{8cm}  
\epsfig{file=deckblatt/Logo-HsRm-1.eps,width=6cm}
\end{minipage}
\hfill
\raisebox{4.3ex}{\small \begin{tabular}[t]{l}
{\textbf{Fachbereich}} \\
Ingenieurwissenschaften \\[1ex]
{\textbf{Studienbereich}} \\
Informationstechnologie und Elektrotechnik \\[1ex]
{\textbf{Studiengang}} \\ 
Elektrotechnik \\[1ex]
{\textbf{Studienrichtung}} \\ 
Elektrotechnik \& Informationstechnik 
\end{tabular}}
\vspace{30mm}
\begin{center}{\Huge\bf Bachelor Thesis} \par
\vspace{20mm}
{\LARGE\bf  \Thema} \par
\vspace{16mm}
{\LARGE\bf  \Bearbeiter} \par
\end{center}
% 
\vspace{15mm}
\begin{flushright}
    Vorgelegt am (Stempel des Studienbereichs): \\
    \vspace{20mm} % Wenn mehr Platz nötig ist
    \rule[0ex]{\textwidth}{0.4pt} \hspace{5mm} (Ort / Datum) \hspace{30ex} (Unterschrift Student)
\end{flushright}
\vfill
\begin{flushleft}
\begin{tabular}{ll}
    Referent:	 &  \Referent \\[0.5ex]
    Korreferent:	 &  \Korreferent
\end{tabular}        
\end{flushleft}
%%%%%%%%%%%%%%%%%%%%%%%%%%%%%%%%%%%%%%%%%%%%
% Blankoseite
\newpage
\thispagestyle{empty}
\rule[0ex]{0ex}{0ex} 	% unsichtbare Markierung, damit Leerseite möglich
%%%%%%%%%%%%%%%%%%%%%%%%%%%%%%%%%%%%%%%%%%%%
% Versicherung
\newpage
\thispagestyle{empty} 
%
{\bf Versicherung}
\par
Hiermit versichere ich, dass ich die vorliegende Arbeit selbst\"andig und ohne unzul\"assige Hilfe Dritter verfasst 
habe.
\par
Die aus fremden Quellen direkt oder indirekt \"ubernommenen Texte, Gedankeng\"ange, Konzepte usw.\ in meinen 
Ausf\"uhrungen habe ich als solche eindeutig gekennzeichnet und mit vollst\"andigen Verweisen auf die jeweilige 
Urheberschaft und Quelle versehen.
\par
Alle weiteren Inhalte wie Textteile, Abbildungen, Tabellen etc.\ ohne entsprechende Verweise stammen im 
urheberrechtlichen Sinn von mir.
\par
Die vorliegende Arbeit wurde bisher weder im In- noch im Ausland in gleicher oder \"ahnlicher Form einer anderen 
Pr\"ufungsbeh\"orde vorgelegt.
\par
Mir ist bekannt, dass ein T\"auschungsversuch vorliegt, wenn sich eine der vorstehenden Versicherungen als 
unrichtig erweist.
\par
\vspace{25mm}
\rule[0ex]{\textwidth}{0.4pt}
(Ort / Datum)\hspace{30ex}	(Unterschrift Student)
%%%%%%%%%%%%%%%%%%%%%%%%%%%%%%%%%%%%%%%%%%%%
% Erklärung zur Einsicht in die Arbeit
\newpage
\thispagestyle{empty} 
%
Der Einsicht in die Bachelor Thesis  und der Ausleihe eines Exemplars der Thesis stimme ich zu / stimme ich nicht zu*.
\par\vspace{10mm}
\rule[0ex]{\textwidth}{0.4pt}
(Ort / Datum)\hspace{30ex}	(Unterschrift Studentin/Student)
\par\vspace{10mm}
Die Zustimmung kann nur bei Vorliegen eines berechtigten Interesses (z.B.\ laufende Forschungsprojekte / Patentschutz) 
verweigert werden. 
\par
Begr\"undung (bei Verweigerung): \\[5ex]
\rule[0ex]{\textwidth}{0.4pt} \\[2ex]
\rule[0ex]{\textwidth}{0.4pt} \\[2ex]
\rule[0ex]{\textwidth}{0.4pt} \\[2ex]
\rule[0ex]{\textwidth}{0.4pt} \\[2ex]
\rule[0ex]{\textwidth}{0.4pt} \\[5ex]
Nur vom Betreuer auszuf\"ullen: \\[2ex]
Gegen die Einsicht in die Bachelor Thesis  und gegen die Ausleihe eines Exemplars der Thesis wird / kein* Einspruch 
erhoben.
\par \vspace{10mm}
\rule[0ex]{\textwidth}{0.4pt}\\
(Ort / Datum)\hspace{30ex} (Unterschrift Betreuer)
\par
Begr\"undung (bei Einspruch): \\[5ex]
\rule[0ex]{\textwidth}{0.4pt} \\[2ex]
\rule[0ex]{\textwidth}{0.0pt} \\[2ex]
\rule[0ex]{\textwidth}{0.4pt} \\[2ex]
\rule[0ex]{\textwidth}{0.4pt} \\[5ex]
*Nichtzutreffendes bitte streichen
%%%%%%%%%%%%%%%%%%%%%%%%%%%%%%%%%%%%%%%%%%%%
%\end{document}
%%%%%%%%%%%%%%%%%%%%%%%%%%%%%%%%%%%%%%%%%%%%
