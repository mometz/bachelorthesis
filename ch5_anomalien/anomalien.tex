\chapter{Anomaliedetektion}\label{ch:anomaliedetektion}
Anomaliedetektion beschreibt die Aufgabe, Trends, Muster und Punkte in einem Datensatz zu finden, die nicht dem Normalzustand
entsprechen~\cite{Chandola2009}. Anders gesagt lautet das Ziel: die Punkte finden, die sich von den anderen Punkten im Datensatz
stark unterscheiden~\cite[Kap.~10]{Tan2014}. Diese andersartigen Datenpunkte oder -sequenzen werden in der Regel als Anomalie,
Ausreißer oder Ausnahmen bezeichnet, wobei Anomalie der geläufigste Begriff ist. Anomaliedetektion findet große Verwendung in
verschiedenen Anwendungsbereichen, wie z.~B.~in der Netzwerktechnik zur Erkennung von potenziellen Angriffen durch Eindringlinge
in ein Netzwerk anhand von ungewöhnlichem Traffic~\Cite{Bernacki2015}. Auch in der Medizin können nach einem EKG durch
Anomaliedetektion Herzrhythmusstörungen erkannt werden~\cite{Chauhan2015}, genau wie eine Bank ein Interesse an Anomalien im
Kreditkartenverhalten ihrer Kunden hat, um Betrugsfälle zu erkennen~\cite{Jiang2023, CeronmaniSharmila2019}.

Die simpelste Herangehensweise zur Erkennung von Anomalien ist die, dass zuerst definiert wird, welche Punkte im Datensatz normalem
Verhalten entsprechen und alle davon abweichenden Punkte als Anomalie zu kennzeichnen. Doch so einfach die Herangehensweise wirkt,
so anfällig ist sie auch für Fehler. Dabei heben sich einige Herausforderungen hervor.

Zum Einen die Frage, wo genau die Grenze zwischen normalem und anomalem Verhalten liegen soll. Eine Region zu definieren, die jeden 
möglichen normalen Punkt beinhaltet und jedmöglichen anomalen Punkt ausschließt, ist nicht trivial und oft nicht präzise durchführbar.
So ist es durchaus möglich, dass in manchen Fällen anomale Punkte als normal bezeichnet werden, und normale Punkte als anomal, je
nachdem, wo die Grenze liegt.

Es stellt sich ebenfalls die Frage, ob eine Anomalie einer binären Natur unterliegt: Entweder es handelt sich um eine Anomalie oder
einen Normalzustand. Doch die Wahrheit liegt oft in der Mitte. Weicht ein Punkt oder eine Sequenz bereits nur leicht vom Normal ab,
so kann es bereits erste Hinweise auf mögliches zukünftiges anomales Verhalten in einer Zeitserie geben, bevor sich solche Datenpunkte
als Anomalie zeigen. Deshalb ist es hilfreich, charakterisieren zu können, wie weit der Punkt oder die Sequenz
vom Normal abweicht. Diese Charakterisierung kann dabei als \textit{Anomaly Score} bezeichnet werden und beispielsweise eine Dezimalzahl
zwischen 0 und 1 sein.

Normalzustände sind in Zeitserien oft zeitvariant und daher schwer festzuhalten bei einer kontinuierlichen Datenaufzeichnung.
Zudem sind Normalzustände und Abweichungen davon in unterschiedlichen Bereichen auch unterschiedlich signifikant. Während beim
menschlichen Körper eine geringe Abweichung der Körpertemperatur bereits gravierend sein kann, ist die gleiche relative Abweichung
in einer anderen Domäne wie in einem Aktienkurs weniger drastisch und unterliegt dementsprechend auch einem Anpassungsbedarf, bevor
es an die Erkennung möglicher Anomalien geht.

Daraus lässt sich direkt zum nächsten Problem übergehen. Die Unterscheidung zwischen globalen und lokalen Anomalien~\cite{Breunig2000}.
Hier ist der Kontext wichtig: Eine Person mit einer Körpergröße von mehr als 2 $m$ ist in ihrer Nachbarschaft sicherlich eine Anomalie,
während sie in einem Basketballteam kaum herausragt~-~im wahrsten Sinne des Wortes. Diese Art der Anomalie wird auch als kontextuelle
Anomalie bezeichnet~\Cite[S.~12]{Wenig2024}.

Doch bevor eine Auswahl an geeigneten Verfahren oder Algorithmen zur Anomaliedetektion getroffen wird, muss zuerst verstanden werden,
welche verschiedenen Arten von Anomalien, im Weiteren als Kategorien bezeichnet, es gibt und wie sich diese voneinander unterscheiden.
Auch wenn Studien zeigen, dass es durchaus Algorithmen gibt, die über mehrere verschiedene Kategorien gut
abschneiden~\cite[S.~30~-~31]{Wenig2024}~\cite{Schmidl2022}, so soll zunächst für jede Kategorie mindestens ein passender Kandidat
gefunden werden. Diese werden dann in einem nächsten Schritt kreuzweise getestet, um auch solche Allrounder entdecken zu können. Dabei
ist auch immer der Kontext der Anwendung wichtig. Wie eingangs erwähnt, sind für verschiedene Tätigkeitsfelder verschiedene
Anforderungen an die Präzision oder Genauigkeit gestellt, weshalb immer die spezifischen Anforderung bedacht werden müssen, und nicht
jeder Algorithmus gleich performant ist über mehrere Datensätze hinweg.

Für die Kategorien wird sich zunächst auf wenige, für diese Arbeit relevante, beschränkt: \textbf{Punkt\-anomalien},
\textbf{Subsequenzanomalien} und \textbf{Korrelationsanomalien}, abgeleitet von Chandola et al.~\cite{Chandola2009}.

\section{Punktanomalien}
Ein einzelner Datenpunkt, der stark von den anderen Punkten im Datensatz abweicht, heißt Punkt\-anomalie~\cite{Chandola2009}. Genauer
gesagt, wenn ein Datenpunkt weit außerhalb der Wahrscheinlichkeitsverteilung des Datensatzes liegt, ist er anomal~\Cite[Kap.~10]{Tan2014}.
Punktanomalien können recht leicht erkannt werden, da Punktanomalien stark vom Mittelwert und vom Median des Datensatzes abweichen. Wenn
von Ausreißern gesprochen wird, sind damit typischerweise Punktanomalien gemeint.

Als Beispielszenario dient ein Smart Meter, das den stündlichen Stromverbrauch misst.
In~\hyperref[subfig:smartmeter]{Abb.~\Ref*{subfig:smartmeter}} ist der gemessene Stromverbrauch dargestellt mit einer klar
erkennbaren Punktanomalie am 01.08.~um 18 Uhr. Die Anomalie wird mit bloßem Auge deutlich und kann auch mit statistischen Größen
nachgewiesen werden, wie in~\hyperref[subfig:smartmeter_histogramm]{Abb.~\Ref*{subfig:smartmeter_histogramm}} anhand der
Häufigkeitsverteilung und dem Mittelwert sowie dem Median zu sehen ist. Das Histogramm dient als gute Approximation für die
Wahrscheinlichkeitsverteilung der Messwerte, und zeigt entsprechend die Eindeutigkeit des Ausreißers.

\begin{figure}[!t]
    \centering
    \begin{subfigure}[b]{0.49\linewidth}
        \includegraphics[width=\linewidth]{ch5_anomalien/abbildungen/punktanomalie_bsp.pdf}
        \caption{Stündliche Smart Meter Messdaten}\label{subfig:smartmeter}
    \end{subfigure}
    \begin{subfigure}[b]{0.49\linewidth}
        \includegraphics[width=\linewidth]{ch5_anomalien/abbildungen/punktanomalie_hist.pdf}
        \caption{\centering Histogramm des gemessenen Stromverbrauchs}\label{subfig:smartmeter_histogramm}
    \end{subfigure}
    \caption{\centering Beispielszenario einer Punktanomalie: Stromverbrauch eines Haushaltes über den Zeitraum von
    drei Tagen. Anhand des Histogramms wird die Anomalie verdeutlicht.}\label{fig:punktanomalie}
\end{figure}

Um nun eine Aussage treffen zu können, ist es wichtig den Kontext der vorliegenden Daten zu kennen. Wenn Daten für ein weitaus größeres
Zeitfenster vorliegen, z.~B.~für eine Woche oder einen Monat, könnte sich möglicherweise zeigen, dass der hohe Verbrauch öfter und
regelmäßiger vorkommt als im gezeigten Zeitraum von drei Tagen. Ob eine globale oder lediglich eine lokale Anomalie vorliegt, wird
mit einem größeren Datensatz besser erkennbar. Die Anomalie könnte beispielsweise auf das gelegentliche Betreiben einer Sauna im Haus
zurückführbar sein, dann würde es sich lediglich um eine lokale Anomalie handeln und in einem größeren Zeitraum in bestimmten Abständen
öfter vorkommen, und wäre somit keine globale Anomalie~\Cite[Kap.~10]{Tan2014}.

Punktanomalien sind im Kontext dieser Arbeit tendenziell weniger relevant, sollen aber aufgrund ihrer grundsätzlichen Bedeutung bzgl.
Anomaliedetektion als einfachste Kategorie trotzdem beleuchtet werden, um entsprechende Algorithmen, die der Erkennung solcher
Punktanomalien zuzuordnen sind, auch gegenüber anderen Anomalien zu testen.

\section{Subsequenzanomalien}

Eine Zeitserie wird gem.~\hyperref[eq:timeseries_set]{Gl.~\Ref*{eq:timeseries_set}} bereits als eine Menge definiert. Demnach wird eine
Subsequenz $S_{i,\,j} = \{\,S_i,\,\dots,\,S_j\,\}\,\subseteq\,S$ von der Zeitserie $S$ umfasst, mit der Länge oder Mächtigkeit
$|\,S_{i,\,j}\,|=j-i+1$ und $|\,S_{i,j}\,|\,\ge\,1$~\cite{Schmidl2022}. Subsequenzanomalien sind Muster in Zeitreihen, die von anderen
Mustern innerhalb der gleichen Zeitreihe abweichen~\cite{Chandola2009}\Cite[S.~12]{Wenig2024}. Im Gegensatz zu Punktanomalien beziehen
sich Subsequenz\-ano\-malien auf mehrere konsekutive Datenpunkte, die ein ungewöhnliches Muster bilden. Eine anomale Subsequenz kann also
bedeuten, dass die Datenpunkte innerhalb der Subsequenz Werte in einem normalen, zu erwartenden Bereich annehmen, aber der zu Grunde
liegende Trend ungewöhnlich ist~\cite{Chandola2009}\cite[S.~17]{Boniol2021}. Solche ungewöhnlichen oder einzigartigen Trends und
Entwicklungen können auf zukünftig auftretende Probleme hindeuten, die sonst unentdeckt bleiben würden.