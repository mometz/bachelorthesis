\chapter{Zielsetzung}
Eine effiziente Ressourcennutzung ist für Unternehmen von zentraler Bedeutung. Produktionsprobleme können dabei enorme finanzielle
Verluste verursachen, wie das Beispiel von Volkswagen im Jahr 2016 zeigt: Durch Störungen im Fertigungsprozess entstand ein wöchentlicher
Schaden von bis zu 400 Millionen Euro~\cite{Krupitzer2020}. Zudem belegen Studien, dass je nach Branche zwischen 15 und 70~\% der
Produktionskosten auf wartungstechnische Ursachen zurückzuführen sind~\cite{You2010}.

Jedoch liegt der Fokus dieser Arbeit nicht auf wirtschaftlichen Faktoren, trotzdem ist zu erwähnen, dass neue, effiziente
Wartungsstrategien basierend auf der Überwachung wichtiger Systemparameter insgesamt eine wirtschaftlichere, kostengünstigere
Alternative zu klassischen Wartungsstrategien %(vgl.~\hyperref[sec:trad_maintenance]{Abs.~\Ref*{sec:trad_maintenance}})
darstellen~\cite{Deloux2009}.

Der Fokus liegt vielmehr in der Entwicklung, Erprobung und Umsetzung der zugrundeliegenden technischen Fragestellung: \textbf{Wie können
Unsupervised Learning Algorithmen zur Anomaliedetektion genutzt werden, um ein effizientes Wartungssystem zu entwickeln?}

Alle weiteren, impliziten Konsequenzen, wie eine Reduzierung der Downtime, Verbesserung der Systemzuverlässigkeit
und Gerätelaufzeit oder die Redundanz eines großen Ersatzteillagers~\cite{Abdelli2022, Mobley2002, Scheffer2004}
wollen im beschränkten Rahmen dieser Arbeit nicht betrachtet werden.