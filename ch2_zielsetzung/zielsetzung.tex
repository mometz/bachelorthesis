\chapter{Zielsetzung}\label{ch:zielsetzung}
Eine effiziente Ressourcennutzung ist für Unternehmen von zentraler Bedeutung. Produktionsprobleme können dabei enorme finanzielle
Verluste verursachen, wie das Beispiel von Volkswagen im Jahr 2016 zeigt: Durch Störungen im Fertigungsprozess entstand ein wöchentlicher
Schaden von bis zu 400 Millionen Euro~\cite{Krupitzer2020}. Zudem zeigen Studien, dass je nach Branche zwischen 15 und 70~\% der
Produktionskosten auf wartungstechnische Ursachen zurückzuführen sind~\cite{Bevilacqua2000}.

Jedoch liegt der Fokus dieser Arbeit nicht auf wirtschaftlichen Faktoren, trotzdem ist zu erwähnen, dass neue, effiziente
Wartungsstrategien basierend auf der Überwachung wichtiger Systemparameter insgesamt eine wirtschaftlichere und kostengünstigere
Alternative zu klassischen Wartungsstrategien (vgl.~\hyperref[sec:trad_maintenance]{Abs.~\Ref*{sec:trad_maintenance}})
darstellen~\cite{Deloux2009}~\Cite[S.~64--65]{Mobley2002}.

Der Fokus liegt vielmehr in der Entwicklung, Erprobung und Umsetzung der zugrundeliegenden technischen Fragestellung: \textbf{Wie können
Unsupervised Learning Algorithmen zur Anomaliedetektion genutzt werden, um zur Entwicklung effizientes Wartungssystem beizutragen?}

Alle weiteren, impliziten Konsequenzen, wie eine Reduzierung der Downtime, Verbesserung der Systemzuverlässigkeit
und Gerätelaufzeit, eine Verbesserung der Produktqualität oder die Redundanz eines großen
Ersatzteillagers~\cite[S.~61--62]{Mobley2002}~\Cite[S.~5]{Scheffer2004} wollen im Rahmen dieser Arbeit nicht betrachtet werden.

Die starke Eingrenzung der Zielsetzung erfolgt auch aufgrund des gegebenen zeitlichen Rahmens der Arbeit von zehn Wochen. Das primäre
Ziel ist also vielmehr die Erprobung verschiedener, sich als potenziell geeignet heraustellender, Machine Learning Algorithmen zur
Anomaliedetektion. Dabei gibt es aufgrund der sich zum Teil sehr grundlegend unterscheidenden Arten von Anomalien entsprechend pro
Anomalietyp separate Kandidaten. Die voneinander abweichenden Anomaliearten und die dafür jeweils nominierten Algorithmen werden in
\hyperref[ch:anomaliedetektion]{Kap.~\Ref*{ch:anomaliedetektion}} vorgestellt und erläutert.

Zur Bewertung und Einordnung der Ergebnisse der Algorithmen ist auch eine Evaluierungsmethode notwendig. Allerdings soll diese nicht
vollautomatisiert sein, sondern nach dem sog.~\textit{Human-in-the-Loop} Ansatz geschehen. So kann die Expertise geschulter und
erfahrener Mitarbeitenden in die Evaluierung miteingebunden werden, zur besseren Beurteilung und Korrektur der Ergebnisse~\cite{Deng2024}.

Schlussendlich und ausblickend soll diese Arbeit einen Beitrag zu besser getimten Wartungseinsätzen beitragen~-~nicht nur für die SSPX1,
sondern für viele, weitere Systeme der Vitronic Produktfamilie, da die zu Grunde liegenden, analysierten Parameter nicht SSPX1-exklusiv
sind.